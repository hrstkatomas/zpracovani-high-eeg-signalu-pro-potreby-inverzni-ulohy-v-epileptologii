%!TEX ROOT=../../Diplomka.tex

\chapter{Závěr}
Epilepsie je mozkové onemocnění projevující se výpadky normální mozkové aktivity. Způsob léčby antiepileptiky je zaměřen spíše na léčbu projevů, než na odstranění příčiny onemocnění. Závažný problém nastává v případě farmakorezistentních pacientů, kdy léky účinkují nedostatečně nebo vůbec. Pro některé pacienty je možnou léčbou chirurgické odstranění epileptogenní zóny, tuto zónu je však nutné správně lokalizovat. Teorie určování zdroje aktivity v mozku z elektroencefalografických nebo magnetoencefalografických dat prodělala v posledních dekádách rozmach. Byly vyvinuty mnohé algoritmy, které se snaží zdroje lokalizovat pomocí různých předpokladů, založených na anatomických, fyziologických a biofyzikálních znalostech. Tyto algoritmy jsou souhrnně nazývány inverzními úlohami a umožňují lokalizovat zdroje aktivity neinvazivně, ze skalpového EEG. Z rozšíření inverzních úloh do klinické praxe by mohlo profitovat mnoho epileptických pacientů. Tato motivace mě vedla k~tomu, abych se v rámci své diplomové práce seznámil s teorií epilepsie, přímých a inverzních úloh, a získané znalosti následně aplikoval při vytváření nástrojů, které by pracovníkům nemocnice Motol umožnily jednoduše aplikovat inverzní úlohu.

Vytvořil jsem nadstavbu Matlab toolboxu SPM12, kterou jsem pojmenoval SPM Motol toolbox. Software umožňuje jednoduché využití inverzních úloh. Výpočet začíná automatizovaným předzpracováním, které se skládá z~opravy reference zesilovačů, filtrace nevhodných frekvenčních pásem, odstranění izolinie, selekce časových oken okolo událostí v~datech a~jejich následného průměrování a vytvoření datového souboru kompatibilního s SPM12. Následuje vytvoření modelu pacientovy hlavy, koregistrace elektrod s MRI a~aplikace inverzní úlohy podle výběru uživatele. Celý tento proces je možné spustit jedinou funkcí \texttt{InversionStart}, která v sobě volá další funkce obsluhující jednotlivé kroky. Výsledky inverze je možné generovat do skleněného mozku, do modelu mozku nebo přímo do MRI snímků. 

Aby bylo možné udělat si představu o praktické přesnosti výsledků dostupných algoritmů inverzních úloh, vygeneroval jsem syntetická data se známou polohou zdroje aktivity, která k tomuto účelu posloužila. Z poloh výsledků algoritmů jsem vypočetl průměrné chyby lokalizace. Nejvyšší přesnosti dosáhl algoritmus EBB, který ovšem občas potlačuje některá ložiska. Naopak nejhorších výsledků dosáhl algoritmus IID kvůli své jednoduchosti a vlastnosti promítat zdroje aktivity k povrchu mozku.

Zpracoval jsem měření somatosenzorických evokovaných potenciálů u pěti testovaných pacientů pro ověření aplikovatelnosti inverzních úloh na reálná data a použitelnost v praxi. Výsledky pacientů P99, P109, P110 a P113 vyšly přesně v očekávaném místě, v gyrus postcentralis kontralaterálně ke stimulované končetině, u pacienta P114 se odezva na evokované potenciály objevila o několik centimetrů posunutá směrem k čelnímu laloku.

V závěru jsem zpracoval kazuistiku epileptického pacienta P81. Bylo lokalizováno potenciální epileptogenní ložisko interiktálních výbojů ve frontálním laloku levé hemisféry. Výsledky inverzní úlohy korelují s předpoklady o~poloze ložiska, která byly získány během dalších analýz.

Mezi omezení SPM Motol toolboxu patří způsob, kterým vytváří modely mozku. Modely jsou vytvářeny geometrickou transformací standardního MNI mozku, ten je však průměrem snímků pacientů s celým mozkem. Toolbox je tedy nevhodný pro zpracovávání případů pacientů s již resekovanými mozky, u kterých ale záchvaty přetrvaly. Model hlavy takového pacienta bude deformovaný a nebude tak splňovat předpoklady pro úspěšnou aplikaci inverzního problému.

Popsané omezení by mohlo být odstraněno během následujícího vývoje SPM Motol toolboxu implementací nového způsobu vytváření modelů hlav pacientů. SPM Motol toolbox je také možné rozšířit o další existující algoritmy inverzních úloh, software také dává prostor pro vývoj zcela nových algoritmů inverzních úloh.

V současnosti SPM Motol toolbox úspěšně využívá Ing. Petr Ježdík, Ph.D. při zpracovávání kazuistik epileptických pacientů nemocnice Motol, toolbox však může využít každý, kdo má přístup k prostředí Matlab. Doufám, že se software bude postupně rozšiřovat do dalších nemocnic a pomáhat epileptickým pacientům nejenom v Motole.